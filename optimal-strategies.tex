\documentclass[12pt]{article}

\usepackage{amssymb}
\usepackage{amsthm, amsmath}
\usepackage{graphicx}
\usepackage{natbib}
\usepackage{color}


\title{Optimal strategies from the explore-exploit trade-off}
\author{Matthew Allcock - University of Sheffield}
\date{}

\begin{document}
\maketitle

\newcommand{\EV}{\mathbb{E}}

\section{Modelling the trade-off}



\section{Optimal stopping if you know everything about the distribution}



\section{Optimal stopping if you know the distribution type}
Lets say you know that the draws are independent and identically distributed and drawn from a distribution with known type but unknown parameters. For example, we might know that the distribution type is normal, so that $\forall n \in \mathbb{N}, X_n \sim N(\mu, \sigma)$, where $\mu$ and $\sigma$ are unknown.

Consider the decision after $n$ draws, where we need to decide whether we should take the next draw or not. To make this decision, we must take the following steps:
\begin{enumerate}
	\item Fit parameters $\mu$ and $\sigma$ to the draws already taken.
	\item Calculate $E$, the expected increase in score from drawing.
	\item Calculate $V$, the value of information that you expect the next draw to give you about the distribution parameters.
	\item If $E + V > c$, where $c$ is the cost of drawing, then draw.
\end{enumerate}

\subsection{Without value of information}
If you follow this rule, you will, on average, stop drawing before you should, \textit{i.e.} before you would if you included the value on information that the extra draw would give you.




\subsection{With value of information}



\section{Simulating the trade-off}


\section{Conclusion}


%\bibliographystyle{unsrt}
%\bibliography{Bibliography}

\end{document}
